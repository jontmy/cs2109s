\subsection{Backpropagation}
Let's first consider a single neuron (perceptron):
\begin{align*}
    & \hat{y} = F(w, x, b) = wx + b \\
    & \varepsilon = L(\hat{y}, y)    
\end{align*}
Now, we want to find $\PartialDerivative{\varepsilon}{w}$ and $\PartialDerivative{\varepsilon}{b}$.

We get the following gradients by applying the chain rule:
\begin{align*}
    & \PartialDerivative{\varepsilon}{w}
    = \PartialDerivative{\varepsilon}{\hat{y}} \cdot \PartialDerivative{\hat{y}}{w} \\
    & \PartialDerivative{\varepsilon}{b}
    = \PartialDerivative{\varepsilon}{\hat{y}} \cdot \PartialDerivative{\hat{y}}{b}
\end{align*}
We don't have any downstream or upstream gradients, since we're working with a singular neuron.

When we use a typical non-linear activation function $\sigma$ in $\hat{y} = \sigma(wx + b)$,
we can split into two parts:
\begin{align*}
    & z = wx + b \\
    & \hat{y} = \sigma(z)
\end{align*}
and then get the gradients by following the computation graph:
\begin{align*}
    & \PartialDerivative{\varepsilon}{w}
    = \PartialDerivative{\varepsilon}{\hat{y}}
    \cdot \PartialDerivative{\hat{y}}{z}
    \cdot \PartialDerivative{z}{w} \\
    & \PartialDerivative{\varepsilon}{b}
    = \PartialDerivative{\varepsilon}{\hat{y}}
    \cdot \PartialDerivative{\hat{y}}{z}
    \cdot \PartialDerivative{z}{b}
\end{align*}

Now let's generalize it to a single layer. Suppose we have an input $x$ with $n$ features
$x_1, x_2, \dots, x_n$, and each feature has its associated weight $w_i$.

Therefore, $x$ is an $n \times 1$ vector and $w$ is a $d \times n$ matrix, where $d$
is the \textbf{embedding dimension}.

Let's add another layer $Z_1$ and consider the original layer as $\hat{Y}$:
\begin{align*}
    & z^{[1]} = (W^{[1]})^T x + b^{[1]} \\
    & \hat{Y} = (W^{[2]})^T z^{[1]} + b^{[2]} \\
    & \varepsilon = L(\hat{Y}, y) = \hat{Y} - Y
\end{align*}

Now, we need to compute two gradients: $\PartialDerivative{\varepsilon}{W^{[1]}}$ and $\PartialDerivative{\varepsilon}{W^{[2]}}$.\\
$\varepsilon$ is a vector, and $W_i$ are matrices.
Suppose we have some arbitary matrices $x$, $W^{[1]}$, and $W^{[2]}$ as follows:
\begin{align*}
    z_1 & = (W_1)^T x \\
    & = \left[ \begin{matrix}
        w^{[1]}_{11} & w^{[1]}_{21} \\
        w^{[1]}_{12} & w^{[1]}_{22} \\
    \end{matrix} \right] \left[
        \begin{matrix}
            x_1 \\
            x_2 \\
        \end{matrix}
    \right] \\
    & = \left[
        \begin{matrix}
            w^{[1]}_{11} x_1 + w^{[1]}_{12} x_2 \\
            w^{[1]}_{21} x_1 + w^{[1]}_{22} x_2 \\
        \end{matrix}
    \right] \\
    \hat{Y} & = (W^{[2]})^T z_1 \\
    & = \left[ \begin{matrix}
        w^{[2]}_{11} & w^{[2]}_{21} \\
        w^{[2]}_{12} & w^{[2]}_{22} \\
    \end{matrix} \right] \left[
        \begin{matrix}
            w^{[1]}_{11} x_1 + w^{[1]}_{12} x_2 \\
            w^{[1]}_{21} x_1 + w^{[1]}_{22} x_2 \\
        \end{matrix}
    \right] \\
    & = \left[
        \begin{matrix}
            w^{[2]}_{11} (w^{[1]}_{11} x_1 + w^{[1]}_{21} x_2) + w^{[2]}_{21} (w^{[1]}_{12} x_1 + w^{[1]}_{22} x_2) \\
            w^{[2]}_{12} (w^{[1]}_{11} x_1 + w^{[1]}_{21} x_2) + w^{[2]}_{22} (w^{[1]}_{12} x_1 + w^{[1]}_{22} x_2) \\
        \end{matrix}
    \right]
\end{align*}
This results in a $2 \times 1$ vector $\hat{Y}$, which we pass to $L$:
\begin{align*}
    e_1 = & \; y_1 \\
    & - w^{[2]}_{11} \left( w^{[1]}_{11} x^{[1]} + w^{[1]}_{21} x^{[2]} \right) \\
    & - w^{[2]}_{12} \left( w^{[1]}_{12} x^{[1]} + w^{[1]}_{22} x^{[2]} \right) \\
    e_2 = & \; y_2 \\
    & - w^{[2]}_{12} \left( w^{[1]}_{11} x^{[1]} + w^{[1]}_{21} x^{[2]} \right) \\
    & - w^{[2]}_{22} \left( w^{[1]}_{12} x^{[1]} + w^{[1]}_{22} x^{[2]} \right) \\
    \implies \varepsilon = & \left[ \begin{matrix}
        e_1 \\
        e_2 \\
    \end{matrix} \right] 
\end{align*}
This entire sequence is \textbf{forward propagation}. Now for backpropagation.

Since our weight matrices are $2 \times 2$, we can calculate $4$ gradients each for
both $\PartialDerivative{e_1}{w^{[2]}}$ and $\PartialDerivative{e_2}{w^{[2]}}$
from $\PartialDerivative{\varepsilon}{w^{[2]}}$ using the chain rule.